\documentclass[a4paper,11pt]{article}
\usepackage[utf8]{inputenc}
\usepackage[spanish,mexico]{babel}
\usepackage[T1]{fontenc}
\usepackage{lmodern}
\usepackage{graphicx}
\usepackage{amsmath, amssymb}
\usepackage{geometry}
\usepackage{hyperref}
\usepackage{fancyhdr}
\usepackage{float}
\usepackage{parskip}
\usepackage{xcolor}
\usepackage{listings}
\usepackage[final]{microtype}

% Configuración de hipervínculos
\hypersetup{
    colorlinks=true,
    linkcolor=blue,
    urlcolor=cyan,
}

% Configuración de márgenes
\geometry{top=2.5cm, bottom=2.5cm, left=2.5cm, right=2.5cm}

% Configuración de encabezado y pie de página
\pagestyle{fancy}
\fancyhf{}
\rhead{\small Manual de Usuario - Visualizador de Arreglos}
\lhead{\small ITBA - 22.21 Electromagnetismo}
\rfoot{\small Página \thepage}

\title{
    \vspace{-2cm}
    \textbf{\Huge Manual de Usuario} \\[0.5cm]
    \textbf{\Large Visualizador de Arreglos de Antenas}
}
\author{
    \textbf{Trabajo Especial - 22.21 Electromagnetismo} \\[1ex]
    Instituto Tecnológico de Buenos Aires (ITBA) \\[0.5cm]
    \textit{Autores:} \\
    Alejandro Nahuel Heir \\
    María de Guadalupe Voss \\[0.5cm]
    {\small Código Fuente (GitHub):}\\
    {\small \url{https://github.com/alheir/AntennaArrayVisualizer}}
}
\date{\today}

\begin{document}

\maketitle
\thispagestyle{empty}

\begin{abstract}
\noindent Este documento describe el funcionamiento, base teórica e instalación del software \textit{Antenna Array Visualizer}. La aplicación permite el diseño y análisis interactivo de arreglos lineales de antenas, soportando configuraciones uniformes y no uniformes, con visualización en tiempo real de diagramas polares, cartesianos y métricas de directividad. Desarrollado en Python utilizando las librerías \texttt{CustomTkinter} y \texttt{Matplotlib}.
\end{abstract}

\tableofcontents
\newpage

\section{Introducción}
El objetivo de este software es proporcionar una herramienta gráfica para el estudio del teorema de multiplicación de diagramas. Este programa permite modificar parámetros como la fase progresiva ($\beta$) y la distribución de corrientes ($I_n$) en tiempo real, facilitando la comprensión de fenómenos como el \textit{beam steering} (direccionamiento del haz) y el \textit{beam shaping} (conformado del haz).

\section{Fundamentos Teóricos y Algoritmos}

\subsection{Principio de Funcionamiento}
El campo total radiado se calcula aplicando el Teorema de Multiplicación de Diagramas:
\begin{equation}
E_{total}(\theta, \phi) = E_{elemento}(\theta, \phi) \times AF(\psi)
\end{equation}

El software implementa algoritmos numéricos para calcular estos factores basándose en la geometría seleccionada por el usuario (Arreglo sobre el eje Z o sobre el eje X).

\subsection{Factor de Arreglo (AF)}
Para un arreglo lineal de $N$ elementos, la respuesta del arreglo está dada por:
\begin{equation}
AF = \sum_{n=0}^{N-1} I_n e^{j(k d_n \cos \gamma + \beta n)}
\end{equation}
Donde $\gamma$ es el ángulo entre el eje del arreglo y el vector de radiación. El software discretiza este cálculo según la orientación:

\subsubsection{Orientación Vertical (Eje Z)}
El arreglo se ubica sobre el eje Z. El corte de elevación (plano XZ) depende de $\theta$:
\begin{equation}
\psi_z = k d \cos(\theta) + \beta
\end{equation}

\subsubsection{Orientación Horizontal (Eje X)}
El arreglo se ubica sobre el eje X.
\begin{itemize}
    \item \textbf{Corte Azimutal (XY):} Depende de $\phi$ (con $\theta=90^\circ$): $\psi_x = k d \cos(\phi) + \beta$.
    \item \textbf{Corte de Elevación (XZ):} Depende de $\theta$ (con $\phi=0^\circ$): $\psi_x = k d \sin(\theta) + \beta$.
\end{itemize}

\subsection{Elementos Radiantes}
Se modelan tres tipos de elementos (asumiendo alineación vertical de los dipolos):
\begin{itemize}
    \item \textbf{Isotrópica:} $F(\theta) = 1$.
    \item \textbf{Dipolo de $\lambda/2$:} $F(\theta) = \frac{\cos(\frac{\pi}{2}\cos\theta)}{\sin\theta}$.
    \item \textbf{Monopolo de $\lambda/4$:} Igual al dipolo, pero nulo para $\theta > \pi/2$ (tierra infinita).
\end{itemize}

\subsection{Cálculo de Directividad y HPBW}
La directividad máxima ($D_{max}$) no se estima, sino que se calcula mediante integración numérica (Regla del Trapecio) del patrón de potencia normalizado $U(\theta, \phi)$:

\begin{equation}
D_{max} = \frac{4\pi U_{max}}{P_{rad}} \quad \text{donde} \quad P_{rad} = \iint_{\Omega} U(\theta, \phi) d\Omega
\end{equation}

El software reporta el valor en dBi ($10\log_{10} D_{linear}$) y el Ancho de Haz de Media Potencia (HPBW) encontrando numéricamente los puntos de -3dB respecto al máximo.

\newpage

\section{Instalación}

\subsection{Requisitos del Sistema}
\begin{itemize}
    \item Sistema Operativo: Windows 10/11 (64 bits).
    \item Espacio en disco: 150 MB aproximadamente.
\end{itemize}

\subsection{Instrucciones}
El software se distribuye con un instalador automatizado generado mediante Inno Setup.

\begin{enumerate}
    \item Ejecute el archivo \texttt{Installer\_AntennaArrayVisualizer.exe}.
    \item Siga las instrucciones del asistente para seleccionar la carpeta de destino.
    \item Al finalizar, podrá iniciar la aplicación desde el acceso directo ``Antenna Array Visualizer'' en el escritorio o menú de inicio.
\end{enumerate}

\section{Uso}

\subsection{Interfaz de Usuario (GUI)}
La ventana principal se divide en el panel de parámetros (izquierda) y el área de visualización (derecha).

\begin{figure}[H]
    \centering
    \fbox{\begin{minipage}{0.9\textwidth}
        \centering
        \includegraphics[width=\textwidth]{gui-overview.png}
    \end{minipage}}
    \caption{Interfaz principal mostrando controles y gráfico polar.}
    \label{fig:gui}
\end{figure}

\subsection{Configuración de Parámetros}
Cada campo de entrada cuenta con \textbf{Tooltips} interactivos: al pasar el mouse sobre la etiqueta del parámetro, aparecerá una ventana emergente con ayuda contextual.

\begin{itemize}
    \item \textbf{N Antennas:} Cantidad de elementos.
    \item \textbf{Separation ($d/\lambda$):} Distancia eléctrica entre elementos.
    \item \textbf{Phase ($\beta$):} Fase progresiva en grados. Modifique este valor para observar el escaneo del haz.
    \item \textbf{Intensities:} Control de amplitudes.
    \begin{itemize}
        \item Valor único (ej. "1"): Arreglo Uniforme.
        \item Lista (ej. "1, 0.5, 0.5, 1"): Arreglo No Uniforme.
    \end{itemize}
    \item \textbf{Type:} Elemento base (Isotrópica, Dipolo, Monopolo).
    \item \textbf{Array Axis:} Define si el arreglo se coloca sobre el eje Z (Vertical) o X (Horizontal).
\end{itemize}

\subsection{Visualización y Herramientas}

\subsubsection{Modos de Vista}
\begin{itemize}
    \item \textbf{View:} Permite alternar entre el corte Vertical (XZ), Horizontal (XY) o \textbf{Both} (Ambos simultáneamente).
    \item \textbf{Plot:} Coordenadas Polares o Cartesianas (dB vs Grados).
    \item \textbf{3D Orientation:} Una casilla de verificación activa un gráfico 3D incrustado en la esquina inferior que muestra la orientación física del arreglo y el plano de corte actual en tiempo real.
\end{itemize}

\subsubsection{Cursor Interactivo}
El software incluye un cursor de medicióno:
\begin{itemize}
    \item \textbf{Hover:} Al pasar el mouse sobre la curva, se muestra el ángulo, la ganancia en dB y la directividad local.
    \item \textbf{Click (Fijar):} Al hacer clic en el gráfico, el cursor se ``congela'' en esa posición, permitiendo analizar un punto específico mientras se modifican otros parámetros. Haga clic nuevamente para liberar.
\end{itemize}

\subsubsection{Métricas}
En el título del gráfico se calculan y muestran automáticamente:
\begin{itemize}
    \item $D_{max}$: Directividad máxima del conjunto en dBi.
    \item $HPBW$: Ancho de haz de media potencia en grados.
    \item $D(^\circ)$: Directividad en la dirección indicada por cursor en dBi.    
\end{itemize}

\section{Bibliografía}
\begin{itemize}
    \item Apuntes de Cátedra, Lic. Patricio A. Marcó, Asignatura 22.21 Electromagnetismo, ITBA.
    \item Balanis, C. A. (2016). \textit{Antenna Theory: Analysis and Design} (4th ed.). John Wiley \& Sons. Caps. 4 y 6.
\end{itemize}

\end{document}